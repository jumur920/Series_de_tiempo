% Options for packages loaded elsewhere
\PassOptionsToPackage{unicode}{hyperref}
\PassOptionsToPackage{hyphens}{url}
%
\documentclass[
]{book}
\usepackage{amsmath,amssymb}
\usepackage{iftex}
\ifPDFTeX
  \usepackage[T1]{fontenc}
  \usepackage[utf8]{inputenc}
  \usepackage{textcomp} % provide euro and other symbols
\else % if luatex or xetex
  \usepackage{unicode-math} % this also loads fontspec
  \defaultfontfeatures{Scale=MatchLowercase}
  \defaultfontfeatures[\rmfamily]{Ligatures=TeX,Scale=1}
\fi
\usepackage{lmodern}
\ifPDFTeX\else
  % xetex/luatex font selection
\fi
% Use upquote if available, for straight quotes in verbatim environments
\IfFileExists{upquote.sty}{\usepackage{upquote}}{}
\IfFileExists{microtype.sty}{% use microtype if available
  \usepackage[]{microtype}
  \UseMicrotypeSet[protrusion]{basicmath} % disable protrusion for tt fonts
}{}
\makeatletter
\@ifundefined{KOMAClassName}{% if non-KOMA class
  \IfFileExists{parskip.sty}{%
    \usepackage{parskip}
  }{% else
    \setlength{\parindent}{0pt}
    \setlength{\parskip}{6pt plus 2pt minus 1pt}}
}{% if KOMA class
  \KOMAoptions{parskip=half}}
\makeatother
\usepackage{xcolor}
\usepackage{longtable,booktabs,array}
\usepackage{calc} % for calculating minipage widths
% Correct order of tables after \paragraph or \subparagraph
\usepackage{etoolbox}
\makeatletter
\patchcmd\longtable{\par}{\if@noskipsec\mbox{}\fi\par}{}{}
\makeatother
% Allow footnotes in longtable head/foot
\IfFileExists{footnotehyper.sty}{\usepackage{footnotehyper}}{\usepackage{footnote}}
\makesavenoteenv{longtable}
\usepackage{graphicx}
\makeatletter
\def\maxwidth{\ifdim\Gin@nat@width>\linewidth\linewidth\else\Gin@nat@width\fi}
\def\maxheight{\ifdim\Gin@nat@height>\textheight\textheight\else\Gin@nat@height\fi}
\makeatother
% Scale images if necessary, so that they will not overflow the page
% margins by default, and it is still possible to overwrite the defaults
% using explicit options in \includegraphics[width, height, ...]{}
\setkeys{Gin}{width=\maxwidth,height=\maxheight,keepaspectratio}
% Set default figure placement to htbp
\makeatletter
\def\fps@figure{htbp}
\makeatother
\setlength{\emergencystretch}{3em} % prevent overfull lines
\providecommand{\tightlist}{%
  \setlength{\itemsep}{0pt}\setlength{\parskip}{0pt}}
\setcounter{secnumdepth}{5}
\usepackage{booktabs}
\ifLuaTeX
  \usepackage{selnolig}  % disable illegal ligatures
\fi
\usepackage[]{natbib}
\bibliographystyle{plainnat}
\usepackage{bookmark}
\IfFileExists{xurl.sty}{\usepackage{xurl}}{} % add URL line breaks if available
\urlstyle{same}
\hypersetup{
  pdftitle={Pronóstico de Ventas de Café en Máquinas Expendedoras},
  pdfauthor={Luisa Angélica Isaza Sanabria - Juan Andrés Murillo Cadena - Carlos Fabián Villa Infante},
  hidelinks,
  pdfcreator={LaTeX via pandoc}}

\title{Pronóstico de Ventas de Café en Máquinas Expendedoras}
\author{Luisa Angélica Isaza Sanabria - Juan Andrés Murillo Cadena - Carlos Fabián Villa Infante}
\date{2025-04-13}

\begin{document}
\maketitle

{
\setcounter{tocdepth}{1}
\tableofcontents
}
\chapter{Introducción}\label{introducciuxf3n}

Durante este curso de series de tiempo, hemos decidido trabajar las ventas de café en una máquina expendedora. Detrás de cada café que alguien compra, hay patrones de consumo, hábitos y decisiones que se repiten en el tiempo. Analizar esta información nos permite aplicar modelos de pronóstico reales, útiles y con impacto directo en la toma de decisiones comerciales y operativas.

Poder anticipar cuánto café se va a vender en los próximos días, semanas o meses es clave para mejorar la experiencia del cliente, reducir pérdidas y aumentar la eficiencia.

\section{Justificación de la Elección}\label{justificaciuxf3n-de-la-elecciuxf3n}

El café es una de las bebidas más consumidas en todo el mundo, y las máquinas expendedoras son una forma práctica de acceder a él. Nos pareció un caso ideal porque:

\begin{verbatim}
1.  Ayuda a planificar mejor los inventarios: Prever la demanda permite tener siempre lo justo: ni mucho producto que termine vencido, ni tan poco que dejemos de vender.

2.  Hace más eficientes las operaciones: Si sabemos cuándo se vende más café, podemos organizar mejor las recargas y los mantenimientos, ahorrando tiempo y dinero.

3.  Permite personalizar promociones:Detectar días u horarios de baja demanda ayuda a lanzar promociones en momentos estratégicos.

4.  Mejora la experiencia de quienes compran: Asegurar que los productos favoritos estén disponibles en los momentos clave mejora la satisfacción y fideliza al cliente.
\end{verbatim}

\section{Descripción de la Información a Utilizar}\label{descripciuxf3n-de-la-informaciuxf3n-a-utilizar}

Vamos a utilizar un dataset llamado ``Coffee Sales'', publicado por Yaroslav Isaienkov en la plataforma Kaggle. Esta base de datos contiene registros reales de ventas desde marzo de 2024 y sigue actualizándose semanalmente.

El dataset incluye:

\begin{verbatim}
•   La fecha y hora de cada transacción
•   El tipo de café vendido
•   La cantidad y el método de pago
•   Información detallada sobre el producto y las preferencias del cliente
\end{verbatim}

Todo esto en archivos están en formato .csv que son muy fáciles de trabajar y analizar, ideales para aplicar modelos de series de tiempo.

\section{Fuentes y Permisos de Uso}\label{fuentes-y-permisos-de-uso}

Una gran ventaja de este dataset es que está disponible bajo una licencia de dominio público (CC0),(\citet{isaienkov2025}) lo que significa que se puede usar libremente con fines educativos, de análisis y sin restricciones. Además, todos los datos han sido recolectados de forma anónima a partir de informes de la propia máquina expendedora.

  \bibliography{book.bib,packages.bib}

\end{document}
